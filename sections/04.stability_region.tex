\subsection{AF relaying with end-to-end ARQ}

Given that $p_{e,i} = 0.1, i = \{1,2 \dots r+1\}$ and $\mu_{AF} = \mu_{MS} = 2$,
the arrival rates at each relay station and at the mobile station are:

\begin{align*}
  \lambda_1 &= \dfrac{\lambda}{(1-0.1)^{r+1}} = \lambda\ 0.9^{-r-1} \\
  \lambda_2 &= \lambda_1 (1-0.1) = \lambda\ 0.9^{-r} \\
  \lambda_3 &= \lambda_2 (1-0.1) = \lambda\ 0.9^{-r + 1} \\
  & \dots \\
  \lambda_{r+1} &= \lambda\ 0.9^{-1}
\end{align*}

Since (a) the entire network is stable when all queues are stable, (b) every
station is a M/M/1 system, and (c) such a system is considered to be stable
when $\rho = \dfrac{\lambda}{\mu} < 1$, then the network is stable if all of the
following inequalities hold:

\begin{align*}
  &\lambda\cdot  0.9^{-r-1} < 2 \\
  &\lambda\cdot  0.9^{-r} < 2 \\
  &\lambda\cdot  0.9^{-r + 1} < 2 \\
  & \dots \\
  &\lambda\cdot  0.9^{-1} < 2
\end{align*}

or

\begin{align*}
  &\lambda < 2\cdot  0.9^{r+1}\\
  &\lambda < 2\cdot   0.9^{r}\\
  &\lambda < 2\cdot   0.9^{r-1}\\
  & \dots \\
  &\lambda < 2\cdot  0.9^{1}
\end{align*}

From these conditions, since $r \geq 1$ and $0.9 < 1$, we can see that the most
restrictive condition is the first one, i.e. if the first one holds, then,
inevitably, all the following conditions will be true. Figure
\ref{fig:04_stability_region_af_e2e} plots function
$\lambda_{max}(r) = 2 \cdot 0.9^{r+1}$ for $1 \leq r \leq 8$. The stability region
is defined as the region where $\lambda < \lambda_{max}(r)$.

\begin{figure}[H]\centering
  % This file was created by matlab2tikz.
%
%The latest updates can be retrieved from
%  http://www.mathworks.com/matlabcentral/fileexchange/22022-matlab2tikz-matlab2tikz
%where you can also make suggestions and rate matlab2tikz.
%
\definecolor{mycolor1}{rgb}{0.00000,0.44700,0.74100}%
%
\begin{tikzpicture}

\begin{axis}[%
width=4.133in,
height=3.26in,
at={(0.693in,0.44in)},
scale only axis,
xmin=1,
xmax=8,
xmajorgrids,
ymin=0.7,
ymax=1.7,
ymajorgrids,
axis background/.style={fill=white}
]
\addplot [color=mycolor1,solid,forget plot]
  table[row sep=crcr]{%
1	1.62\\
2	1.458\\
3	1.3122\\
4	1.18098\\
5	1.062882\\
6	0.9565938\\
7	0.86093442\\
8	0.774840978\\
};
\end{axis}
\end{tikzpicture}%
  \caption{Function $\lambda_{max}(r) = 2\cdot  0.9^{r+1}$ for $1 \leq r \leq 8$.
    The stability region is the area under the curve.}
  \label{fig:04_stability_region_af_e2e}
\end{figure}

\subsection{AF relaying with end-to-end ARQ}
Given that $p_{e,i} = 0.1$, $i = \{1,2 \dots r+1\}$, $r=4$ and
$\mu_{AF} = \mu_{MS} = 2$, the average end-to-end delay is

\begin{align}
  T = \dfrac{5}{2 \cdot 0.9^5 - \lambda}
  \label{eq:05_T_af_e2e}
\end{align}

In order for the network to be stable,
$0 \leq \lambda \leq 2 \cdot 0.9^5 = 1.181$, for $r=4$. Figure
\ref{fig:05_arrival_rate_af_e2e} illustrates how the end-to-end delay changes
in relation to the arrival rate $\lambda$.

\begin{figure}\centering
  % This file was created by matlab2tikz.
%
%The latest updates can be retrieved from
%  http://www.mathworks.com/matlabcentral/fileexchange/22022-matlab2tikz-matlab2tikz
%where you can also make suggestions and rate matlab2tikz.
%
\definecolor{mycolor1}{rgb}{0.00000,0.44700,0.74100}%
%
\begin{tikzpicture}

\begin{axis}[%
width=4.133in,
height=3.26in,
at={(0.693in,0.44in)},
scale only axis,
xmin=0,
xmax=1.2,
xmajorgrids,
xlabel={$\lambda$: arrival rate},
ymin=0,
ymax=400,
ymajorgrids,
ylabel={T: average end-to-end delay},
axis background/.style={fill=white}
]
\addplot [color=mycolor1,solid,forget plot]
  table[row sep=crcr]{%
0	4.23377195210757\\
0.0119191919191919	4.27693749156493\\
0.0238383838383838	4.3209922883818\\
0.0357575757575758	4.36596410807058\\
0.0476767676767677	4.41188188420691\\
0.0595959595959596	4.45877578050645\\
0.0715151515151515	4.506677256903\\
0.0834343434343434	4.55561913993224\\
0.0953535353535353	4.60563569775195\\
0.107272727272727	4.65676272015904\\
0.119191919191919	4.709037603996\\
0.131111111111111	4.76249944437506\\
0.143030303030303	4.81718913218776\\
0.154949494949495	4.87314945841097\\
0.166868686868687	4.93042522576865\\
0.178787878787879	4.9890633683616\\
0.190707070707071	5.04911307993654\\
0.202626262626263	5.11062595153144\\
0.214545454545455	5.17365611930639\\
0.226464646464646	5.23826042345039\\
0.238383838383838	5.30449857914451\\
0.25030303030303	5.3724333606622\\
0.262222222222222	5.44213079979973\\
0.274141414141414	5.51366039995535\\
0.286060606060606	5.58709536731597\\
0.297979797979798	5.66251286076784\\
0.30989898989899	5.73999426232493\\
0.321818181818182	5.81962547006702\\
0.333737373737374	5.90149721580476\\
0.345656565656566	5.98570540994101\\
0.357575757575758	6.07235151628457\\
0.369494949494949	6.1615429598957\\
0.381414141414141	6.25339357141034\\
0.393333333333333	6.34802407170728\\
0.405252525252525	6.44556260125718\\
0.417171717171717	6.54614529903453\\
0.429090909090909	6.64991693649208\\
0.441010101010101	6.75703161280653\\
0.452929292929293	6.86765351841683\\
0.464848484848485	6.98195777481197\\
0.476767676767677	7.100131359602\\
0.488686868686869	7.22237412715055\\
0.500606060606061	7.34889993648769\\
0.512525252525252	7.47993789989334\\
0.524444444444444	7.61573376748657\\
0.536363636363636	7.75655146542411\\
0.548282828282828	7.902674807965\\
0.56020202020202	8.05440940676915\\
0.572121212121212	8.21208480445782\\
0.584040404040404	8.37605686378095\\
0.595959595959596	8.54671044884561\\
0.607878787878788	8.72446244092481\\
0.61979797979798	8.90976513859095\\
0.631717171717172	9.1031101005535\\
0.643636363636364	9.30503249994078\\
0.655555555555555	9.51611607123976\\
0.667474747474747	9.73699874618929\\
0.679393939393939	9.9683790932279\\
0.691313131313131	10.2110236974137\\
0.703232323232323	10.4657756450618\\
0.715151515151515	10.7335643109637\\
0.727070707070707	11.0154166876219\\
0.738989898989899	11.312470547583\\
0.750909090909091	11.6259897944948\\
0.762828282828283	11.9573824396055\\
0.774747474747475	12.3082217429337\\
0.786666666666667	12.6802711887332\\
0.798585858585859	13.0755141318572\\
0.810505050505051	13.4961891669498\\
0.822424242424242	13.944832552141\\
0.834343434343434	14.4243293852438\\
0.846262626262626	14.937975714171\\
0.858181818181818	15.4895544077639\\
0.87010101010101	16.0834284800803\\
0.882020202020202	16.7246567390906\\
0.893939393939394	17.4191382488382\\
0.905858585858586	18.1737943431403\\
0.917777777777778	18.9968000945618\\
0.92969696969697	19.8978816594592\\
0.941616161616162	20.8887024613221\\
0.953535353535353	21.9833708012872\\
0.965454545454545	23.199115902783\\
0.977373737373737	24.5572014117166\\
0.989292929292929	26.0841797078782\\
1.00121212121212	27.8136452057704\\
1.01313131313131	29.7887346828733\\
1.0250505050505	32.0657743528219\\
1.0369696969697	34.7197380658791\\
1.04888888888889	37.8526606214565\\
1.06080808080808	41.6070579018947\\
1.07272727272727	46.1882127681014\\
1.08464646464646	51.903005341291\\
1.09656565656566	59.2316399865022\\
1.10848484848485	68.9701296638436\\
1.12040404040404	82.540995361029\\
1.13232323232323	102.760627940095\\
1.14424242424242	136.100433871686\\
1.15616161616162	201.463561550169\\
1.16808080808081	387.621180560988\\
1.18	5102.04081632547\\
};
\end{axis}
\end{tikzpicture}%

  \caption{The relation between end-to-end delay and the arrival rate in a
    queueing network with AF relaying and end-to-end ARQ with $r=4$.}
  \label{fig:05_arrival_rate_af_e2e}
\end{figure}

If $p_{e,i}$ increases, then $\lambda_{max}$ decreases, as
$\lambda_{max} = 2 \cdot (1-p_{e,i})^5$, and the law that governs the average
delay remains the same. Thus, as $p_{e,i}$ increases, the arrival rate-delay
curve is shifted towards the left.


%-------------------------------------------------------------------------------
\subsection{DF relaying with end-to-end ARQ}

Given that $p_{e,i} = 0.1, i = \{1,2 \dots r+1\}$, $r=4$ and
$\mu_{MS} = 2\mu_{DF} = 2$, the average end-to-end delay is

\begin{align}
  T = \dfrac{1}{1.8 - \lambda} + \sum\limits_{i=1}^4 \dfrac{1}{0.9^{6-i} - \lambda}
  \label{eq:05_T_df_e2e}
\end{align}


In order for the network to be stable,
$0 \leq \lambda \leq 0.9^5 = 0.5905$, for $r=4$. Figure
\ref{fig:05_arrival_rate_df_e2e} illustrates how the end-to-end delay changes in
relation to the arrival rate $\lambda$.


\begin{figure}\centering
  % This file was created by matlab2tikz.
%
%The latest updates can be retrieved from
%  http://www.mathworks.com/matlabcentral/fileexchange/22022-matlab2tikz-matlab2tikz
%where you can also make suggestions and rate matlab2tikz.
%
\definecolor{mycolor1}{rgb}{0.00000,0.44700,0.74100}%
%
\begin{tikzpicture}

\begin{axis}[%
width=4.133in,
height=3.26in,
at={(0.693in,0.44in)},
scale only axis,
xmin=0,
xmax=0.6,
xmajorgrids,
ymin=0,
ymax=2500,
ymajorgrids,
axis background/.style={fill=white}
]
\addplot [color=mycolor1,solid,forget plot]
  table[row sep=crcr]{%
0	6.37953225287473\\
0.00595959595959596	6.43307250414471\\
0.0119191919191919	6.48756386897636\\
0.0178787878787879	6.54303262836284\\
0.0238383838383838	6.59950605848217\\
0.0297979797979798	6.65701247892595\\
0.0357575757575758	6.71558130379402\\
0.0417171717171717	6.77524309585805\\
0.0476767676767677	6.83602962401342\\
0.0536363636363636	6.89797392425733\\
0.0595959595959596	6.96111036445094\\
0.0655555555555555	7.02547471314548\\
0.0715151515151515	7.09110421277632\\
0.0774747474747475	7.15803765755551\\
0.0834343434343434	7.22631547642262\\
0.0893939393939394	7.29597982144567\\
0.0953535353535353	7.36707466209956\\
0.101313131313131	7.43964588588836\\
0.107272727272727	7.51374140582126\\
0.113232323232323	7.58941127529964\\
0.119191919191919	7.66670781102579\\
0.125151515151515	7.74568572460262\\
0.131111111111111	7.82640226355904\\
0.137070707070707	7.90891736260812\\
0.143030303030303	7.99329380602621\\
0.148989898989899	8.07959740213139\\
0.154949494949495	8.16789717094011\\
0.160909090909091	8.25826554619398\\
0.166868686868687	8.35077859307434\\
0.172828282828283	8.44551624306436\\
0.178787878787879	8.54256254757716\\
0.184747474747475	8.64200595214785\\
0.190707070707071	8.74393959318894\\
0.196666666666667	8.84846161953725\\
0.202626262626263	8.95567554127783\\
0.208585858585859	9.06569060862377\\
0.214545454545455	9.17862222396233\\
0.22050505050505	9.29459239055658\\
0.226464646464646	9.41373020182283\\
0.232424242424242	9.53617237559674\\
0.238383838383838	9.66206383836581\\
0.244343434343434	9.79155836509345\\
0.25030303030303	9.92481928100527\\
0.256262626262626	10.0620202325677\\
0.262222222222222	10.2033460358828\\
0.268181818181818	10.3489936118742\\
0.274141414141414	10.4991730189789\\
0.28010101010101	10.6541085956175\\
0.286060606060606	10.8140402265401\\
0.292020202020202	10.9792247492795\\
0.297979797979798	11.1499375194584\\
0.303939393939394	11.3264741566581\\
0.30989898989899	11.5091524960629\\
0.315858585858586	11.698314775262\\
0.321818181818182	11.8943300905513\\
0.327777777777778	12.0975971630183\\
0.333737373737374	12.3085474618259\\
0.33969696969697	12.5276487407138\\
0.345656565656566	12.7554090541563\\
0.351616161616162	12.9923813322934\\
0.357575757575758	13.2391686092507\\
0.363535353535353	13.4964300185005\\
0.369494949494949	13.7648876924276\\
0.375454545454545	14.0453347324534\\
0.381414141414141	14.3386444525169\\
0.387373737373737	14.6457811445026\\
0.393333333333333	14.9678126720933\\
0.399292929292929	15.3059252732032\\
0.405252525252525	15.6614410455943\\
0.411212121212121	16.0358387122612\\
0.417171717171717	16.4307784219731\\
0.423131313131313	16.848131548877\\
0.429090909090909	17.2900167313268\\
0.435050505050505	17.7588437596841\\
0.441010101010101	18.257367422372\\
0.446969696969697	18.7887541021805\\
0.452929292929293	19.3566648590356\\
0.458888888888889	19.9653600581543\\
0.464848484848485	20.6198324812496\\
0.470808080808081	21.3259785673051\\
0.476767676767677	22.0908213993886\\
0.482727272727273	22.9228049766831\\
0.488686868686869	23.8321883210114\\
0.494646464646465	24.8315819718243\\
0.500606060606061	25.9366917154832\\
0.506565656565657	27.167370829517\\
0.512525252525252	28.5491434658752\\
0.518484848484848	30.1154685928051\\
0.524444444444444	31.9112070593894\\
0.53040404040404	33.9981192895743\\
0.536363636363636	36.4639466790656\\
0.542323232323232	39.4381615639411\\
0.548282828282828	43.1209459620268\\
0.554242424242424	47.8405715425955\\
0.56020202020202	54.1782059639913\\
0.566161616161616	63.2755886490463\\
0.572121212121212	77.7405248356129\\
0.578080808080808	105.159077652149\\
0.584040404040404	181.072314167341\\
0.59	2068.5110650018\\
};
\end{axis}
\end{tikzpicture}%
  \caption{The relation between end-to-end delay and the arrival rate in a
    queueing network with DF relaying and end-to-end ARQ with $r=4$.}
  \label{fig:05_arrival_rate_df_e2e}
\end{figure}

If $p_{e,i}$ increases, then $\lambda_{max}$ decreases, as
$\lambda_{max} = (1-p_{e,i})^5$, and the law that governs the average
delay remains the same. Thus, as $p_{e,i}$ increases, the arrival rate-delay
curve is shifted towards the left.


%-------------------------------------------------------------------------------
\subsection{DF relaying with hop-by-hop ARQ}

Given that $p_{e,i} = 0.1, i = \{1,2 \dots r+1\}$, $r=4$ and
$\mu_{MS} = 2\mu_{DF} = 2$, the average end-to-end delay is

\begin{align*}
  T &= \dfrac{1 - p_{e,r+1}}{\mu_{MS}(1 - p_{e,r+1}) - \lambda} + \sum\limits_{i=1}^r \dfrac{1-p_{e,i}}{\mu_{DF} (1 - p_{e,i}) - \lambda} \\
  ~ &= \dfrac{0.9}{1.8 - \lambda} + \sum\limits_{i=1}^4 \dfrac{0.9}{0.9 - \lambda} \\
  ~ &= \dfrac{0.9}{1.8 - \lambda} + \dfrac{3.6}{0.9 - \lambda} \\
\end{align*}

In order for the network to be stable,
$0 \leq \lambda \leq 0.9$, for $r=4$. Figure
\ref{fig:05_arrival_rate_df_hbh} illustrates how the end-to-end delay changes in
relation to the arrival rate $\lambda$.

\begin{figure}\centering
  % This file was created by matlab2tikz.
%
%The latest updates can be retrieved from
%  http://www.mathworks.com/matlabcentral/fileexchange/22022-matlab2tikz-matlab2tikz
%where you can also make suggestions and rate matlab2tikz.
%
\definecolor{mycolor1}{rgb}{0.00000,0.44700,0.74100}%
%
\begin{tikzpicture}

\begin{axis}[%
width=4.133in,
height=3.26in,
at={(0.693in,0.44in)},
scale only axis,
unbounded coords=jump,
xmin=0,
xmax=1,
xmajorgrids,
ymin=0,
ymax=400,
ymajorgrids,
axis background/.style={fill=white}
]
\addplot [color=mycolor1,solid,forget plot]
  table[row sep=crcr]{%
0	4.5\\
0.00909090909090909	4.5433543975966\\
0.0181818181818182	4.58757626762045\\
0.0272727272727273	4.63269230769231\\
0.0363636363636364	4.67873033098209\\
0.0454545454545455	4.72571932532246\\
0.0545454545454545	4.77368951612903\\
0.0636363636363636	4.8226724334168\\
0.0727272727272727	4.8727009832273\\
0.0818181818181818	4.92380952380952\\
0.0909090909090909	4.97603394692804\\
0.1	5.02941176470588\\
0.109090909090909	5.08398220244716\\
0.118181818181818	5.13978629792583\\
0.127272727272727	5.19686700767263\\
0.136363636363636	5.25526932084309\\
0.145454545454545	5.31504038130544\\
0.154545454545455	5.37622961864978\\
0.163636363636364	5.43888888888889\\
0.172727272727273	5.50307262569832\\
0.181818181818182	5.568838003129\\
0.190909090909091	5.63624511082138\\
0.2	5.70535714285714\\
0.209090909090909	5.77624060150376\\
0.218181818181818	5.84896551724138\\
0.227272727272727	5.92360568661147\\
0.236363636363636	6.00023892959541\\
0.245454545454545	6.07894736842105\\
0.254545454545455	6.15981772990886\\
0.263636363636364	6.2429416737109\\
0.272727272727273	6.32841614906832\\
0.281818181818182	6.41634378302219\\
0.290909090909091	6.50683330336271\\
0.3	6.6\\
0.309090909090909	6.69596622889306\\
0.318181818181818	6.79486196319018\\
0.327272727272727	6.8968253968254\\
0.336363636363636	7.00200360649169\\
0.345454545454545	7.11055327868853\\
0.354545454545455	7.22264150943396\\
0.363636363636364	7.33844668526067\\
0.372727272727273	7.45815945530419\\
0.381818181818182	7.58198380566802\\
0.390909090909091	7.71013824884793\\
0.4	7.84285714285714\\
0.409090909090909	7.98039215686275\\
0.418181818181818	8.12301390268123\\
0.427272727272727	8.27101375445746\\
0.436363636363636	8.42470588235294\\
0.445454545454545	8.58442953020134\\
0.454545454545455	8.75055157198014\\
0.463636363636364	8.9234693877551\\
0.472727272727273	9.10361410667444\\
0.481818181818182	9.29145427286357\\
0.490909090909091	9.4875\\
0.5	9.69230769230769\\
0.509090909090909	9.90648542417294\\
0.518181818181818	10.1306990881459\\
0.527272727272727	10.3656794425087\\
0.536363636363636	10.6122302158273\\
0.545454545454545	10.871237458194\\
0.554545454545455	11.1436803688052\\
0.563636363636364	11.4306438791733\\
0.572727272727273	11.7333333333333\\
0.581818181818182	12.053091684435\\
0.590909090909091	12.3914197257851\\
0.6	12.75\\
0.609090909090909	13.1307251908397\\
0.618181818181818	13.5357320099256\\
0.627272727272727	13.9674418604651\\
0.636363636363636	14.4286099137931\\
0.645454545454545	14.9223847019123\\
0.654545454545455	15.452380952381\\
0.663636363636364	16.0227692307692\\
0.672727272727273	16.6383870967742\\
0.681818181818182	17.3048780487805\\
0.690909090909091	18.0288667141839\\
0.7	18.8181818181818\\
0.709090909090909	19.6821428571429\\
0.718181818181818	20.6319327731092\\
0.727272727272727	21.6810883140054\\
0.736363636363636	22.8461538461539\\
0.745454545454545	24.1475659229209\\
0.754545454545455	25.6108695652174\\
0.763636363636364	27.2684210526316\\
0.772727272727273	29.1618204804045\\
0.781818181818182	31.345467032967\\
0.790909090909091	33.8918918918919\\
0.8	36.9\\
0.809090909090909	40.508256880734\\
0.818181818181818	44.9166666666667\\
0.827272727272727	50.4252336448599\\
0.836363636363636	57.5053908355794\\
0.845454545454545	66.9428571428571\\
0.854545454545455	80.1519230769232\\
0.863636363636364	99.9611650485437\\
0.872727272727273	132.970588235294\\
0.881818181818182	198.980198019802\\
0.890909090909091	396.990000000002\\
0.9	inf\\
};
\end{axis}
\end{tikzpicture}%
  \caption{The relation between end-to-end delay and the arrival rate in a
    queueing network with DF relaying and hop-by-hop ARQ with $r=4$.}
  \label{fig:05_arrival_rate_df_hbh}
\end{figure}

If $p_{e,i}$ increases, then $\lambda_{max}$ decreases, as
$\lambda_{max} = 1-p_{e,i}$, and the law that governs the average
delay remains the same. Thus, as $p_{e,i}$ increases, the arrival rate-delay
curve is shifted towards the left.


Figure \ref{fig:05_arrival_rates} shows the relation between the arrival rate
and the end-to-end delay for all three queueing networks considered.

\begin{figure}[H]\centering
  % This file was created by matlab2tikz.
%
%The latest updates can be retrieved from
%  http://www.mathworks.com/matlabcentral/fileexchange/22022-matlab2tikz-matlab2tikz
%where you can also make suggestions and rate matlab2tikz.
%
\definecolor{mycolor1}{rgb}{0.00000,0.44700,0.74100}%
\definecolor{mycolor2}{rgb}{0.85000,0.32500,0.09800}%
\definecolor{mycolor3}{rgb}{0.92900,0.69400,0.12500}%
%
\begin{tikzpicture}

\begin{axis}[%
width=4.133in,
height=3.26in,
at={(0.693in,0.44in)},
scale only axis,
unbounded coords=jump,
xmin=0,
xmax=1.2,
xmajorgrids,
ymin=0,
ymax=350,
ymajorgrids,
axis background/.style={fill=white}
]
\addplot [color=mycolor1,solid,forget plot]
  table[row sep=crcr]{%
0	4.23377195210757\\
0.0119191919191919	4.27693749156493\\
0.0238383838383838	4.3209922883818\\
0.0357575757575758	4.36596410807058\\
0.0476767676767677	4.41188188420691\\
0.0595959595959596	4.45877578050645\\
0.0715151515151515	4.506677256903\\
0.0834343434343434	4.55561913993224\\
0.0953535353535353	4.60563569775195\\
0.107272727272727	4.65676272015904\\
0.119191919191919	4.709037603996\\
0.131111111111111	4.76249944437506\\
0.143030303030303	4.81718913218776\\
0.154949494949495	4.87314945841097\\
0.166868686868687	4.93042522576865\\
0.178787878787879	4.9890633683616\\
0.190707070707071	5.04911307993654\\
0.202626262626263	5.11062595153144\\
0.214545454545455	5.17365611930639\\
0.226464646464646	5.23826042345039\\
0.238383838383838	5.30449857914451\\
0.25030303030303	5.3724333606622\\
0.262222222222222	5.44213079979973\\
0.274141414141414	5.51366039995535\\
0.286060606060606	5.58709536731597\\
0.297979797979798	5.66251286076784\\
0.30989898989899	5.73999426232493\\
0.321818181818182	5.81962547006702\\
0.333737373737374	5.90149721580476\\
0.345656565656566	5.98570540994101\\
0.357575757575758	6.07235151628457\\
0.369494949494949	6.1615429598957\\
0.381414141414141	6.25339357141034\\
0.393333333333333	6.34802407170728\\
0.405252525252525	6.44556260125718\\
0.417171717171717	6.54614529903453\\
0.429090909090909	6.64991693649208\\
0.441010101010101	6.75703161280653\\
0.452929292929293	6.86765351841683\\
0.464848484848485	6.98195777481197\\
0.476767676767677	7.100131359602\\
0.488686868686869	7.22237412715055\\
0.500606060606061	7.34889993648769\\
0.512525252525252	7.47993789989334\\
0.524444444444444	7.61573376748657\\
0.536363636363636	7.75655146542411\\
0.548282828282828	7.902674807965\\
0.56020202020202	8.05440940676915\\
0.572121212121212	8.21208480445782\\
0.584040404040404	8.37605686378095\\
0.595959595959596	8.54671044884561\\
0.607878787878788	8.72446244092481\\
0.61979797979798	8.90976513859095\\
0.631717171717172	9.1031101005535\\
0.643636363636364	9.30503249994078\\
0.655555555555555	9.51611607123976\\
0.667474747474747	9.73699874618929\\
0.679393939393939	9.9683790932279\\
0.691313131313131	10.2110236974137\\
0.703232323232323	10.4657756450618\\
0.715151515151515	10.7335643109637\\
0.727070707070707	11.0154166876219\\
0.738989898989899	11.312470547583\\
0.750909090909091	11.6259897944948\\
0.762828282828283	11.9573824396055\\
0.774747474747475	12.3082217429337\\
0.786666666666667	12.6802711887332\\
0.798585858585859	13.0755141318572\\
0.810505050505051	13.4961891669498\\
0.822424242424242	13.944832552141\\
0.834343434343434	14.4243293852438\\
0.846262626262626	14.937975714171\\
0.858181818181818	15.4895544077639\\
0.87010101010101	16.0834284800803\\
0.882020202020202	16.7246567390906\\
0.893939393939394	17.4191382488382\\
0.905858585858586	18.1737943431403\\
0.917777777777778	18.9968000945618\\
0.92969696969697	19.8978816594592\\
0.941616161616162	20.8887024613221\\
0.953535353535353	21.9833708012872\\
0.965454545454545	23.199115902783\\
0.977373737373737	24.5572014117166\\
0.989292929292929	26.0841797078782\\
1.00121212121212	27.8136452057704\\
1.01313131313131	29.7887346828733\\
1.0250505050505	32.0657743528219\\
1.0369696969697	34.7197380658791\\
1.04888888888889	37.8526606214565\\
1.06080808080808	41.6070579018947\\
1.07272727272727	46.1882127681014\\
1.08464646464646	51.903005341291\\
1.09656565656566	59.2316399865022\\
1.10848484848485	68.9701296638436\\
1.12040404040404	82.540995361029\\
1.13232323232323	102.760627940095\\
1.14424242424242	136.100433871686\\
1.15616161616162	201.463561550169\\
1.16808080808081	387.621180560988\\
1.18	5102.04081632547\\
};
\addplot [color=mycolor2,solid,forget plot]
  table[row sep=crcr]{%
0	6.37953225287473\\
0.00595959595959596	6.43307250414471\\
0.0119191919191919	6.48756386897636\\
0.0178787878787879	6.54303262836284\\
0.0238383838383838	6.59950605848217\\
0.0297979797979798	6.65701247892595\\
0.0357575757575758	6.71558130379402\\
0.0417171717171717	6.77524309585805\\
0.0476767676767677	6.83602962401342\\
0.0536363636363636	6.89797392425733\\
0.0595959595959596	6.96111036445094\\
0.0655555555555555	7.02547471314548\\
0.0715151515151515	7.09110421277632\\
0.0774747474747475	7.15803765755551\\
0.0834343434343434	7.22631547642262\\
0.0893939393939394	7.29597982144567\\
0.0953535353535353	7.36707466209956\\
0.101313131313131	7.43964588588836\\
0.107272727272727	7.51374140582126\\
0.113232323232323	7.58941127529964\\
0.119191919191919	7.66670781102579\\
0.125151515151515	7.74568572460262\\
0.131111111111111	7.82640226355904\\
0.137070707070707	7.90891736260812\\
0.143030303030303	7.99329380602621\\
0.148989898989899	8.07959740213139\\
0.154949494949495	8.16789717094011\\
0.160909090909091	8.25826554619398\\
0.166868686868687	8.35077859307434\\
0.172828282828283	8.44551624306436\\
0.178787878787879	8.54256254757716\\
0.184747474747475	8.64200595214785\\
0.190707070707071	8.74393959318894\\
0.196666666666667	8.84846161953725\\
0.202626262626263	8.95567554127783\\
0.208585858585859	9.06569060862377\\
0.214545454545455	9.17862222396233\\
0.22050505050505	9.29459239055658\\
0.226464646464646	9.41373020182283\\
0.232424242424242	9.53617237559674\\
0.238383838383838	9.66206383836581\\
0.244343434343434	9.79155836509345\\
0.25030303030303	9.92481928100527\\
0.256262626262626	10.0620202325677\\
0.262222222222222	10.2033460358828\\
0.268181818181818	10.3489936118742\\
0.274141414141414	10.4991730189789\\
0.28010101010101	10.6541085956175\\
0.286060606060606	10.8140402265401\\
0.292020202020202	10.9792247492795\\
0.297979797979798	11.1499375194584\\
0.303939393939394	11.3264741566581\\
0.30989898989899	11.5091524960629\\
0.315858585858586	11.698314775262\\
0.321818181818182	11.8943300905513\\
0.327777777777778	12.0975971630183\\
0.333737373737374	12.3085474618259\\
0.33969696969697	12.5276487407138\\
0.345656565656566	12.7554090541563\\
0.351616161616162	12.9923813322934\\
0.357575757575758	13.2391686092507\\
0.363535353535353	13.4964300185005\\
0.369494949494949	13.7648876924276\\
0.375454545454545	14.0453347324534\\
0.381414141414141	14.3386444525169\\
0.387373737373737	14.6457811445026\\
0.393333333333333	14.9678126720933\\
0.399292929292929	15.3059252732032\\
0.405252525252525	15.6614410455943\\
0.411212121212121	16.0358387122612\\
0.417171717171717	16.4307784219731\\
0.423131313131313	16.848131548877\\
0.429090909090909	17.2900167313268\\
0.435050505050505	17.7588437596841\\
0.441010101010101	18.257367422372\\
0.446969696969697	18.7887541021805\\
0.452929292929293	19.3566648590356\\
0.458888888888889	19.9653600581543\\
0.464848484848485	20.6198324812496\\
0.470808080808081	21.3259785673051\\
0.476767676767677	22.0908213993886\\
0.482727272727273	22.9228049766831\\
0.488686868686869	23.8321883210114\\
0.494646464646465	24.8315819718243\\
0.500606060606061	25.9366917154832\\
0.506565656565657	27.167370829517\\
0.512525252525252	28.5491434658752\\
0.518484848484848	30.1154685928051\\
0.524444444444444	31.9112070593894\\
0.53040404040404	33.9981192895743\\
0.536363636363636	36.4639466790656\\
0.542323232323232	39.4381615639411\\
0.548282828282828	43.1209459620268\\
0.554242424242424	47.8405715425955\\
0.56020202020202	54.1782059639913\\
0.566161616161616	63.2755886490463\\
0.572121212121212	77.7405248356129\\
0.578080808080808	105.159077652149\\
0.584040404040404	181.072314167341\\
0.59	2068.5110650018\\
};
\addplot [color=mycolor3,solid,forget plot]
  table[row sep=crcr]{%
0	4.5\\
0.00909090909090909	4.5433543975966\\
0.0181818181818182	4.58757626762045\\
0.0272727272727273	4.63269230769231\\
0.0363636363636364	4.67873033098209\\
0.0454545454545455	4.72571932532246\\
0.0545454545454545	4.77368951612903\\
0.0636363636363636	4.8226724334168\\
0.0727272727272727	4.8727009832273\\
0.0818181818181818	4.92380952380952\\
0.0909090909090909	4.97603394692804\\
0.1	5.02941176470588\\
0.109090909090909	5.08398220244716\\
0.118181818181818	5.13978629792583\\
0.127272727272727	5.19686700767263\\
0.136363636363636	5.25526932084309\\
0.145454545454545	5.31504038130544\\
0.154545454545455	5.37622961864978\\
0.163636363636364	5.43888888888889\\
0.172727272727273	5.50307262569832\\
0.181818181818182	5.568838003129\\
0.190909090909091	5.63624511082138\\
0.2	5.70535714285714\\
0.209090909090909	5.77624060150376\\
0.218181818181818	5.84896551724138\\
0.227272727272727	5.92360568661147\\
0.236363636363636	6.00023892959541\\
0.245454545454545	6.07894736842105\\
0.254545454545455	6.15981772990886\\
0.263636363636364	6.2429416737109\\
0.272727272727273	6.32841614906832\\
0.281818181818182	6.41634378302219\\
0.290909090909091	6.50683330336271\\
0.3	6.6\\
0.309090909090909	6.69596622889306\\
0.318181818181818	6.79486196319018\\
0.327272727272727	6.8968253968254\\
0.336363636363636	7.00200360649169\\
0.345454545454545	7.11055327868853\\
0.354545454545455	7.22264150943396\\
0.363636363636364	7.33844668526067\\
0.372727272727273	7.45815945530419\\
0.381818181818182	7.58198380566802\\
0.390909090909091	7.71013824884793\\
0.4	7.84285714285714\\
0.409090909090909	7.98039215686275\\
0.418181818181818	8.12301390268123\\
0.427272727272727	8.27101375445746\\
0.436363636363636	8.42470588235294\\
0.445454545454545	8.58442953020134\\
0.454545454545455	8.75055157198014\\
0.463636363636364	8.9234693877551\\
0.472727272727273	9.10361410667444\\
0.481818181818182	9.29145427286357\\
0.490909090909091	9.4875\\
0.5	9.69230769230769\\
0.509090909090909	9.90648542417294\\
0.518181818181818	10.1306990881459\\
0.527272727272727	10.3656794425087\\
0.536363636363636	10.6122302158273\\
0.545454545454545	10.871237458194\\
0.554545454545455	11.1436803688052\\
0.563636363636364	11.4306438791733\\
0.572727272727273	11.7333333333333\\
0.581818181818182	12.053091684435\\
0.590909090909091	12.3914197257851\\
0.6	12.75\\
0.609090909090909	13.1307251908397\\
0.618181818181818	13.5357320099256\\
0.627272727272727	13.9674418604651\\
0.636363636363636	14.4286099137931\\
0.645454545454545	14.9223847019123\\
0.654545454545455	15.452380952381\\
0.663636363636364	16.0227692307692\\
0.672727272727273	16.6383870967742\\
0.681818181818182	17.3048780487805\\
0.690909090909091	18.0288667141839\\
0.7	18.8181818181818\\
0.709090909090909	19.6821428571429\\
0.718181818181818	20.6319327731092\\
0.727272727272727	21.6810883140054\\
0.736363636363636	22.8461538461539\\
0.745454545454545	24.1475659229209\\
0.754545454545455	25.6108695652174\\
0.763636363636364	27.2684210526316\\
0.772727272727273	29.1618204804045\\
0.781818181818182	31.345467032967\\
0.790909090909091	33.8918918918919\\
0.8	36.9\\
0.809090909090909	40.508256880734\\
0.818181818181818	44.9166666666667\\
0.827272727272727	50.4252336448599\\
0.836363636363636	57.5053908355794\\
0.845454545454545	66.9428571428571\\
0.854545454545455	80.1519230769232\\
0.863636363636364	99.9611650485437\\
0.872727272727273	132.970588235294\\
0.881818181818182	198.980198019802\\
0.890909090909091	396.990000000002\\
0.9	inf\\
};
\end{axis}
\end{tikzpicture}%
  \caption{The relation between end-to-end delay and the arrival rate in a
    queueing network with AF relaying with end-to-end AQR (\texttt{blue}),
    DF relaying and end-to-end ARQ (\texttt{red}) and
    DF relaying and hop-by-hop ARQ (\texttt{orange}) for $r=4$.}
  \label{fig:05_arrival_rates}
\end{figure}

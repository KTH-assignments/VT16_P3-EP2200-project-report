\subsection{Packet arrival rate at each DF and at the MS}

For this queueing network, the arrival rate at the first relay station is
$$\lambda_1 = \lambda + \lambda_1 p_{e,1} \Leftrightarrow \lambda_1 = \dfrac{\lambda}{1 - p_{e,1}}$$

From the structure of the queueing network, the arrival at the subsequent relay
stations rate will be the sum of the error-free arrivals from its previous
station and the erroneous packets that were processed at that specific station

\begin{equation}
 \left.
  \begin{array}{rl}
    \lambda_2 &= \lambda_2 p_{e,2} + (1 - p_{e,1}) \lambda_1 \Leftrightarrow
    \lambda_2 (1 - p_{e,2}) = (1 - p_{e,1}) \lambda_1 \\
    \lambda_3 &= \lambda_3 p_{e,3} + (1 - p_{e,2}) \lambda_2 \Leftrightarrow
    \lambda_3 (1 - p_{e,3}) = (1 - p_{e,2}) \lambda_2 \\
    & \dots \\
    \lambda_{r+1} &= \lambda_r p_{e,r} + (1 - p_{e,r}) \lambda_r \Leftrightarrow
    \lambda_{r+1} (1 - p_{e,r+1}) = (1 - p_{e,r}) \lambda_r \\
  \end{array}
\right\} \label{eq:03_1}
\end{equation} \\

From this set of equations we observe that
$$\lambda_1 (1-p_{e,1}) = \lambda_2 (1-p_{e,2}) = \lambda_3 (1-p_{e,3})
= \dots = \lambda_{r+1} (1 - p_{e,r+1})$$

Hence the arrival rates at each relay station $i$ and at the mobile station are

\begin{align*}
  \lambda_i = \dfrac{\lambda}{1-p_{e,i}}
\end{align*}

\begin{align*}
  \lambda_{r+1} = \dfrac{\lambda}{1-p_{e,r+1}}
\end{align*}


% ------------------------------------------------------------------------------
\subsection{Average number of packets}

The offered load at each relay station i and at the mobile station is then given
by equations \ref{eq:03_2} and \ref{eq:03_3}.

\begin{equation}
  \rho_i^{DF} = \lambda_i \overline{x}_i = \dfrac{\lambda}{\mu_{DF} (1 - p_{e,i})}
  \label{eq:03_2}
\end{equation}

\begin{equation}
  \rho^{MS} = \lambda_{r+1} \overline{x}_{r+1} = \dfrac{\lambda}{\mu_{MS} (1 - p_{e,r+1})}
  \label{eq:03_3}
\end{equation}


With these pieces of information, we can derive the average number of packets
in each queue, the average number of packets being serviced, and then, from
Little's theorem, the average queueing delay.

The average number of packets in a queue of a M/M/1 system is
$N_q = \dfrac{\rho^2}{1-\rho}$, hence the average number of packets in the queues
of a relay station $i$ and in the mobile station is given by equations
\ref{eq:03_4} and \ref{eq:03_5} respectively.

\begin{equation}
  N_{i,q}^{DF} = \dfrac{(\rho_i^{DF})^2}{1 - \rho_i^{DF}} = \dfrac{\lambda^2}{\mu_{DF}(1 - p_{e,i})\Big(\mu_{DF}(1 - p_{e,i})- \lambda\Big)}
  \label{eq:03_4}
\end{equation}

\begin{equation}
  N_q^{MS} = \dfrac{(\rho^{MS})^2}{1 - \rho^{MS}} = \dfrac{\lambda^2}{\mu_{MS}(1 - p_{e,r+1})\Big(\mu_{MS}(1 - p_{e,r+1})- \lambda\Big)}
  \label{eq:03_5}
\end{equation}

The average number of packets under service at a relay station $i$ and at the
mobile station is given by equations \ref{eq:03_2} and \ref{eq:03_3} respectively.

Hence, the average number of packets at every relay station $i$ and at the
mobile station are given by equations \ref{eq:02_6} and \ref{eq:02_7}.

\begin{equation}
  N_i^{DF} = \dfrac{\rho_i^{DF}}{1 - \rho_i^{DF}} = \dfrac{\lambda}{\mu_{DF}(1 - p_{e,i}) - \lambda}
  \label{eq:03_6}
\end{equation}

\begin{equation}
  N^{MS} =  \dfrac{\rho^{MS}}{1 - \rho^{MS}} = \dfrac{\lambda}{\mu_{MS}(1 - p_{e,r+1}) - \lambda}
  \label{eq:03_7}
\end{equation}


% ------------------------------------------------------------------------------
\subsection{Average queueing delay}

From Little's theorem, the average queueing delay is $W = \dfrac{N_q}{\lambda}$.
Equations \ref{eq:03_8} and \ref{eq:03_9} give the average queueing delay at a
relay station $i$ and at the mobile station respectively.

\begin{align}
  W_i^{DF} &= \dfrac{N_{i,q}^{DF}}{\lambda_i} =
    \dfrac{\dfrac{\lambda^2}{\mu_{DF}(1 - p_{e,i})\Big(\mu_{DF}(1 - p_{e,i}) - \lambda\Big)}} {\dfrac{\lambda}{1-p_{e,i}}} \nonumber \\
         ~ &= \dfrac{\lambda}{\mu_{DF}\Big(\mu_{DF}(1 - p_{e,i})- \lambda\Big)}
  \label{eq:03_8}
\end{align}

\begin{align}
  W^{MS} &= \dfrac{N_q^{MS}}{\lambda_{r+1}} =
    \dfrac{\dfrac{\lambda^2}{\mu_{MS}(1 - p_{e,r+1})\Big(\mu_{MS}(1 - p_{e,r+1}) - \lambda\Big)}}{\dfrac{\lambda}{1-p_{e,r+1}}} \nonumber \\
       ~ &= \dfrac{\lambda}{\mu_{MS}\Big(\mu_{MS}(1 - p_{e,r+1})- \lambda\Big)}
  \label{eq:03_9}
\end{align}


% ------------------------------------------------------------------------------
\subsection{Average end-to-end delay}

The average delay at a relay station $i$ and at the mobile station is then the
sum of the waiting time and the service time at that station:

\begin{align*}
  T_i^{DF} &= W_i^{DF} + \dfrac{1}{\mu_{DF}} \\
         ~ &= \dfrac{\lambda}{\mu_{DF}\Big(\mu_{DF}(1 - p_{e,i})- \lambda\Big)} + \dfrac{1}{\mu_{DF}} \\
         ~ &= \dfrac{1-p_{e,i}}{\mu_{DF} (1 - p_{e,i}) - \lambda}
\end{align*}

\begin{align*}
  T^{MS} &= W^{MS} + \dfrac{1}{\mu_{MS}} \\
       ~ &= \dfrac{\lambda}{\mu_{MS}\Big(\mu_{MS}(1 - p_{e,r+1})- \lambda\Big)} + \dfrac{1}{\mu_{MS}} \\
       ~ &= \dfrac{1 - p_{e,r+1}}{\mu_{MS}(1 - p_{e,r+1}) - \lambda}
\end{align*}


The average end-to-end delay $T$ is then the sum of all waiting and serving times,
over $r$ relay stations and at the mobile station.

\begin{align*}
  T &= T^{MS} + \sum\limits_{i=1}^r T_i^{DF} \\
  ~ &= \dfrac{1 - p_{e,r+1}}{\mu_{MS}(1 - p_{e,r+1}) - \lambda} + \sum\limits_{i=1}^r \dfrac{1-p_{e,i}}{\mu_{DF} (1 - p_{e,i}) - \lambda}
\end{align*}


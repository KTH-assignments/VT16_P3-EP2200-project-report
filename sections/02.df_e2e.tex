\subsection{Packet arrival rate at each DF and at the MS}

For this queueing network, the different arrival rates for each station are:

\begin{equation}
 \left.
  \begin{array}{rl}
    \lambda_2 &= \lambda_1 (1 - p_{e,1}) \\
    \lambda_3 &= \lambda_2 (1 - p_{e,2}) = \lambda_1 (1 - p_{e,1}) (1 - p_{e,2}) \\
    & \dots \\
    \lambda_{r+1} &= \lambda_r (1-p_{e,r}) = \lambda_1 \prod\limits_{k=1}^{r} (1 - p_{e,k})
  \end{array}
\right\} \label{eq:02_1}
\end{equation} \\

From the structure of the queueing network:

\begin{align*}
  \lambda_1 &= \lambda + \lambda_1 p_{e,1} + \lambda_2 p_{e,2} + \lambda_3 p_{e,3}
  + \dots
  + \lambda_{r+1} p_{e,r+1} \\
  ~ &= \lambda + \lambda_1 p_{e,1} + \lambda_1 (1 - p_{e,1}) p_{e,2}
  + \lambda_1 (1 - p_{e,1})(1 - p_{e,2}) p_{e,3}
  + \dots
  + \lambda_1 \prod\limits_{i=1}^r (1 - p_{e,i}) p_{e,r+1} \\
  ~ &= \lambda + \lambda_1 \sum_{k=0}^r \Big(p_{e,k+1} \prod\limits_{i=1}^k (1 - p_{e,i})\Big)
\end{align*}

Hence

\begin{equation*}
  \lambda_1 = \dfrac{\lambda}{1 - \sum\limits_{k=0}^r \Big(p_{e,k+1} \prod\limits_{i=1}^k (1 - p_{e,i})\Big)}
\end{equation*}

and from the set of equations \ref{eq:02_1}, we obtain the packet arrival rates
$\lambda_i$ and $\lambda_{r+1}$ at every relay station $i$ and at the mobile
station:

\begin{align}
  \lambda_i = \dfrac{\lambda \prod\limits_{k=1}^{i-1} (1 - p_{e,k})}{1 - \sum\limits_{k=0}^r \Big(p_{e,k+1} \prod\limits_{i=1}^k (1 - p_{e,i})\Big)}
  \label{eq:02_lambda_i}
\end{align}

\begin{align}
  \lambda_{r+1} = \dfrac{\lambda \prod\limits_{k=1}^{r} (1 - p_{e,k})}{1 - \sum\limits_{k=0}^r \Big(p_{e,k+1} \prod\limits_{i=1}^k (1 - p_{e,i})\Big)}
  \label{eq:02_lambda_r_plus_one}
\end{align}


% ------------------------------------------------------------------------------
\subsection{Average number of packets}

The offered load at each relay station i and at the mobile station is then given
by equations \ref{eq:02_2} and \ref{eq:02_3}.

\begin{equation}
  \rho_i^{DF} = \lambda_i \overline{x}_{DF} = \dfrac{\lambda_1}{\mu_{DF}} =
  \dfrac{\lambda \prod\limits_{k=1}^{i-1} (1 - p_{e,k})}{\mu_{DF}\Bigg(1 - \sum\limits_{k=0}^r \Big(p_{e,k+1} \prod\limits_{i=1}^k (1 - p_{e,i})\Big)\Bigg)}
  \label{eq:02_2}
\end{equation}

\begin{equation}
  \rho^{MS} = \lambda_{r+1} \overline{x}_{MS} = \dfrac{\lambda_1}{\mu_{MS}} =
  \dfrac{\lambda \prod\limits_{k=1}^{r} (1 - p_{e,k})}{\mu_{DF}\Bigg(1 - \sum\limits_{k=0}^r \Big(p_{e,k+1} \prod\limits_{i=1}^k (1 - p_{e,i})\Big)\Bigg)}
  \label{eq:02_3}
\end{equation}











With these pieces of information, we can derive the average number of packets
in each queue, the average number of packets being serviced, and then, from
Little's theorem, the average queueing delay.

The average number of packets in a queue of a M/M/1 system is
$N_q = \dfrac{\rho^2}{1-\rho}$, hence the average number of packets in the queues
of a relay station $i$ and in the mobile station is given by equations
\ref{eq:02_4} and \ref{eq:02_5} respectively.

\begin{equation}
  N_{i,q}^{DF} = \dfrac{(\rho_i^{DF})^2}{1 - \rho_i^{DF}}
  \label{eq:02_4}
\end{equation}

\begin{equation}
  N_q^{MS} = \dfrac{(\rho^{MS})^2}{1 - \rho^{MS}}
  \label{eq:02_5}
\end{equation}

The average number of packets under service at a relay station $i$ and at the
mobile station is given by equations \ref{eq:02_2} and \ref{eq:02_3} respectively.

Hence, the average number of packets at every relay station $i$ and at the
mobile station are given by equations \ref{eq:02_6} and \ref{eq:02_7}.

\begin{equation}
  N_i^{DF} = \dfrac{\rho_i^{DF}}{1 - \rho_i^{DF}}
  \label{eq:02_6}
\end{equation}

\begin{equation}
  N^{MS} =  \dfrac{\rho^{MS}}{1 - \rho^{MS}}
  \label{eq:02_7}
\end{equation}


% ------------------------------------------------------------------------------
\subsection{Average queueing delay}

From Little's theorem, the average queueing delay is $W = \dfrac{N_q}{\lambda}$.
Equations \ref{eq:02_8} and \ref{eq:02_9} give the average queueing delay at a
relay station $i$ and at the mobile station respectively.

\begin{equation}
  W_i^{DF} = \dfrac{N_{i,q}^{DF}}{\lambda_i}
  \label{eq:02_8}
\end{equation}

\begin{equation}
  W^{MS} = \dfrac{N_q^{MS}}{\lambda_{r+1}}
  \label{eq:02_9}
\end{equation}


% ------------------------------------------------------------------------------
\subsection{Average end-to-end delay}

The average delay at a relay station $i$ and at the mobile station is then the
sum of the waiting time and the service time at that station:

\begin{equation}
  T_i^{DF} = W_i^{DF} + \dfrac{1}{\mu_{DF}}
  \label{eq:02_10}
\end{equation}

\begin{equation}
  T^{MS} = W^{MS} + \dfrac{1}{\mu_{MS}}
  \label{eq:02_11}
\end{equation}


The average end-to-end delay $T$ is then the sum of all waiting and serving times,
over $r$ relay stations and at the mobile station.

\begin{align*}
  T &= T^{MS} + \sum\limits_{i=1}^r T_i^{DF}= \dfrac{N^{MS} + \sum\limits_{i=1}^r N_i^{DF}}{\lambda}
\end{align*}
